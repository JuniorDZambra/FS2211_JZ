\documentclass[a4paper,10pt]{article}

\usepackage[activeacute]{babel}
\usepackage[utf8]{inputenc}
\usepackage{bookman}
\usepackage{color}
\usepackage{graphicx,wrapfig}
\usepackage{anysize}
\usepackage[pdftex=true,colorlinks=true,linkcolor=black,urlcolor=blue,bookmarksopen=true]{hyperref}
\usepackage{bookmark}
\usepackage{amssymb,amsmath,cancel}

\begin{document}

La carga electrica, representada por el simbolo Q o q, es una cantidad escalar que caracteriza los estados de
electricidad de un cuerpo en vitrea (carga positiva) y resinosa (carga negativa). Cuando dos cuerpos con el
mismo estado de electridad (tipo de carga) se juntan, manifiestan una interaccion del tipo repulsiva,
mientras que la interaccion que se produce entre dos cuerpos con el mismo estado de electricidad sera del tipo
atractiva.\\

Las unidades empleadas para la carga electrica son: el Coulomb (C), en el sistema MKS; el Amperio por segundo
($A\cot s$), en el sistema internacional (SI); y el franklin (Fr), en el sistema cegesimal (cgs). La equivalencia
entre las diferentes unidades es la siguiente: $1C = 1A\cdot s=3\times10^{9}Fr$.\\ 

En todo sistema aislado electricamente la carga electrica neta no cambia con el tiempo si no se agrega o se
extrae de el carga de algun modo, es decir, que la cantidad $Q(t)$ (modelo matematico para la carga neta) se
conserva y se cumple la siguiente relacion:\\
\begin{equation*}
    \frac{\mathrm{d}Q(t)}{\mathrm{d}t}=0 \quad\therefore\quad Q(t)=Q(t_{0}),\:\forall t\in\mathbb{R}.
\end{equation*}

\section*{Modelo at\'omico}

Michael Faraday (1791-1867), mientras realizaba experimentos de electrolicis, descubrio que para depositar un
mol de cualquier sustancia en un electrodo era necesaria siempre la misma cantidad de electricidad.\\

Aunque para ese momento se establecio el concepto de carga elemental $e$, su valor no fue conocido hasta
determinado el numero de Avogadro.\\

\section*{Distribuci\'on continua de carga}

La densidad de carga es todo conglomerado de materia que tiene carga y masa, que no puede ser representado
mediante un n\'umero entero finito de carga y masa.

Estos conglomerados de materia pueden distribuirse en una regi\'on limitada del espacio.

Con estos conglomerados distribuidos en el espacio se logra formar una densidad de materia (carga o masa)
a partir de la comparacion de la cantidad de materia contenida en una regi\'on.

Tambi\'en puedo definirla como la comparacion de materia que tengo contenida en una determinada region del
espacio, que pueden ser finitas o infinitesimales.

La densidad de carga puede ser media (finita), densidad promedio de una carga localizada en una determinada regi\'on finita del espacio
o local (infinitesimal), densidad de una carga infinitesimal localizada en una regi\'on infinitesimal del espacio.

La densidad de carga local se determina mediante el limite en que el espacio que tiene contenida una carga
disminuye tanto como se quiera, esto es, el limite de la densida de carga.

El valor de la carga depende de la distribucion, que puede ser uniforme o variable, de la misma en una
determinada region del espacio.

La densidad de carga media se determina mediante el valor medio espacial de la densidad de carga local.

Aunque la carga local sea no uniforme, el valor de la densidad de carga media seguir\'a siendo la misma.

La densidad es una variable intensiva, es decir, si juntamos dos subsistemas de igual densidad en
un solo sistema, este \'ultimo no tendr\'a como densidad total la suma de 

Cuando la carga se distribuye de manera uniforme o de forma homog\'enea, se dice que la densidad local
coincide con la densidad media.

Hay una relaci\'on de proporcionalidad entre el valor de la carga y el tamaño del espacio que la contiene.

Hay tres tipos de densidades, que dependen del tipo de espacio (n\'umero de dimensiones).



\end{document}